\documentclass[10pt,a4paper]{report}
\usepackage[hmargin=2cm,vmargin=3.5cm,bmargin=2cm]{geometry}
\usepackage[utf8]{inputenc}
\usepackage[portuguese]{babel}
\usepackage[T1]{fontenc}
\usepackage{amsmath}
\usepackage{amsfonts}
\usepackage{amssymb}
\usepackage{makeidx}
\usepackage{graphicx}
\usepackage{listings}
\usepackage{indentfirst}
\usepackage[pdftex]{hyperref}
\usepackage{csvsimple}
\usepackage{color}
\usepackage{mathtools}
\usepackage{float}

% Pseudocode
\usepackage{algpseudocode}


% Headers and Footers
\usepackage{fancyhdr}

\definecolor{mygreen}{rgb}{0,0.6,0}
\definecolor{mygray}{rgb}{0.5,0.5,0.5}
\definecolor{mymauve}{rgb}{0.58,0,0.82}

\lstset{ %
  backgroundcolor=\color{white},   % choose the background color; you must add \usepackage{color} or \usepackage{xcolor}
  basicstyle=\footnotesize,        % the size of the fonts that are used for the code
  breakatwhitespace=false,         % sets if automatic breaks should only happen at whitespace
  breaklines=true,                 % sets automatic line breaking
  captionpos=b,                    % sets the caption-position to bottom
  commentstyle=\color{mygreen},    % comment style
  deletekeywords={...},            % if you want to delete keywords from the given language
  escapeinside={\%*}{*)},          % if you want to add LaTeX within your code
  extendedchars=true,              % lets you use non-ASCII characters; for 8-bits encodings only, does not work with UTF-8
  frame=single,                    % adds a frame around the code
  keywordstyle=\color{blue},       % keyword style
%  language=Octave,                 % the language of the code
  morekeywords={*,...},            % if you want to add more keywords to the set
  numbers=left,                    % where to put the line-numbers; possible values are (none, left, right)
  numbersep=5pt,                   % how far the line-numbers are from the code
  numberstyle=\tiny\color{mygray}, % the style that is used for the line-numbers
  rulecolor=\color{black},         % if not set, the frame-color may be changed on line-breaks within not-black text (e.g. comments (green here))
  showspaces=false,                % show spaces everywhere adding particular underscores; it overrides 'showstringspaces'
  showstringspaces=false,          % underline spaces within strings only
  showtabs=false,                  % show tabs within strings adding particular underscores
  stepnumber=2,                    % the step between two line-numbers. If it's 1, each line will be numbered
  stringstyle=\color{mymauve},     % string literal style
  tabsize=2,                       % sets default tabsize to 2 spaces
  title=\lstname                   % show the filename of files included with \lstinputlisting; also try caption instead of title
}

\lstdefinestyle{base}{
  language=C++,
  emptylines=1,
  breaklines=true,
  basicstyle=\ttfamily\color{black},
  moredelim=**[is][\color{blue}]{@}{@},
  basicstyle=\small
}

% Headers and Footers
\pagestyle{fancy} % fancy style
\lhead{\rightmark} % left header
\rhead{\leftmark} % right header
\lfoot{} % left footer
\cfoot{\textbf{\thepage}} % central footer
\rfoot{} % right footer
% Headers and Footers

\makeindex

% define the title
\author{}
\title{MC458 - Laboratório 1: Algoritmos de Ordenação}
\date{September 28, 2013}
\begin{document}

% generates the title
\maketitle

% insert the table of contents
\tableofcontents


\chapter{Métodos}

\section{Improved cache memory use}

One of the greatest optimizations we achieved, was the better use of the cache memory. Some data intensive reading algorithms can benefit from the cache memory, by accessing sequential memory addresses. With lesser cache misses, the algorithm has no need to search the data in the RAM memory, which is much slower than cache memories.


\chapter{Otimizações}
\newpage
\section{Sgemm}

\begin{center}
\begin{minipage}{.48\textwidth}\small
\begin{lstlisting}[caption=Parboil sequential source code for sgemm,style=base]

	for (int mm = 0; mm < m; ++mm) {
		for (int nn = 0; nn < n; ++nn) {
			float c = 0.0f;
			for (int i = 0; i < k; ++i) {
				float a = A[mm + i * lda]; 
				float b = B[nn + i * ldb];
				c += a * b;
			}
			C[mm+nn*ldc] = C[mm+nn*ldc] * beta + alpha * c;
		}
	}
\end{lstlisting}
\end{minipage}
\hfill
\begin{minipage}{.48\textwidth}\small
\begin{lstlisting}[caption=Parboil parallel source code for sgemm,style=base]

	@#pragma omp parallel for collapse (2)@
	for (int mm = 0; mm < m; ++mm) {
		for (int nn = 0; nn < n; ++nn) {
			float c = 0.0f;
			for (int i = 0; i < k; ++i) {
				float a = A[mm + i * lda]; 
				float b = B[nn + i * ldb];
				c += a * b;
			}
			C[mm+nn*ldc] = C[mm+nn*ldc] * beta + alpha * c;
		}
	}
\end{lstlisting}
\end{minipage}
\end{center}

For the matrices multiplication algorithm, the matrices are represented as arrays, so, we improved the memory access by transposing the matrices, this way, in the inner loop, the index i is only added instead of multiplied to calculate the line and column position of the matrix in the array. We needed then to aloccate auxiliary arrays to represent the transposed matrices and then transpose while making the copy from one array to another. \\

To avoid the increase of the execution time of the algorithm, we parallelized this transpositioning. This can be done because it has no loop-carried dependencies and the data is distributed almost equally for all threads without concurrency, in other words, it is high parallelizable. \\

We created the variables aux1 and aux2 because in the inner loop, these values are constant and in serial and parboil parallel versions, these constant is calculated every iteration of the inner loop wasting some time in unnecessary multiplications. The access to the array is almost sequential after this changes. using better the memory, caused some great improvement in execution time of this algoritm.\\

\begin{lstlisting}[caption=Our parallel source code for sgemm,style=base]

	tb = (float *) malloc(ldb * n * sizeof(float));
	@# pragma omp parallel for collapse(2)@
	for (i = 0; i < ldb; i++)
		for (j = 0; j < n; j++)
			tb[j * n + i] = B[i * n + j];

	ta = (float *) malloc(lda * m * sizeof(float));
	@# pragma omp parallel for collapse(2)@
	for (i = 0; i < lda; i++)
		for (j = 0; j < m; j++)
			ta[j * m + i] = A[i * m + j];

	@# pragma omp parallel for private(nn, c, i, a, b, mm) collapse(2) schedule (static)@
	for (mm = 0; mm < m; ++mm) {
		for (nn = 0; nn < n; ++nn) {
			c = 0.0f;
			aux1 = mm * lda;
			aux2 = nn * ldb;
			for (i = 0; i < k; ++i) {
				c += ta[i + aux1] * tb[i + aux2];
			}
			C[mm+nn*ldc] = C[mm+nn*ldc] * beta + alpha * c;
		}
	}
\end{lstlisting}

\newpage
\chapter{Análise e Resultados}

\begin{itemize}
\item Vetor Ordenado
\item Vetor Invertido
\item Vetor Constante
\item Vetor Aleatório
\end{itemize}

\section{Algoritmos $\mathbf{O(n^2)}$}

\subsection{Bubble Sort}

%\begin{figure}[H]
%\includegraphics[width=\textwidth]{Imagens/bubble.png}
%\caption{Bubble Sort para diversos formatos de vetores}
%\end{figure}

\begin{samepage}
\section{Comparações dos Algoritmos}
\begin{table}[H]\small
\caption{Tabela dos tempos para os vetores aleatórios}
\begin{center}
\begin{tabular}{|c|c|c|c|c|c|}
\hline 
N & Bubble Sort (us) & Insertion Sort (us) & Merge Sort (us) & Heap Sort (us) & Quick Sort (us) \\ 
\hline
20 & 0.54 & 0.38 & 2.18 & 0.89 & 0.33 \\
\hline
40 & 2.06 & 1.28 & 4.74 & 2.09 & 0.85 \\
\hline
60 & 5.41 & 4.38 & 2.64 & 3.38 & 1.25 \\
\hline
80 & 5.19 & 6.54 & 3.71 & 4.66 & 2.21 \\
\hline
100 & 6.30 & 5.42 & 4.69 & 4.15 & 0.97 \\
\hline
120 & 9.86 & 5.62 & 5.61 & 2.64 & 1.22 \\
\hline
140 & 13.29 & 8.16 & 7.16 & 3.21 & 1.62 \\
\hline
160 & 18.98 & 10.39 & 7.75 & 3.97 & 1.71 \\
\hline
180 & 24.19 & 13.20 & 8.81 & 4.58 & 1.87 \\
\hline
200 & 31.50 & 17.26 & 9.77 & 5.07 & 2.18 \\
\hline
220 & 39.08 & 18.71 & 10.73 & 5.64 & 2.30 \\
\hline
240 & 46.73 & 24.89 & 11.97 & 6.38 & 2.54 \\
\hline
260 & 55.65 & 26.57 & 13.28 & 6.89 & 3.06 \\
\hline
280 & 62.90 & 32.87 & 14.36 & 7.59 & 3.59 \\
\hline
300 & 71.83 & 36.10 & 15.41 & 8.38 & 3.74 \\
\hline
320 & 82.09 & 38.36 & 16.36 & 8.99 & 3.78 \\
\hline
340 & 89.37 & 48.51 & 17.87 & 9.87 & 4.25 \\
\hline
360 & 103.90 & 52.94 & 19.01 & 10.53 & 4.32 \\
\hline
380 & 114.72 & 57.14 & 20.40 & 11.57 & 5.18 \\
\hline
400 & 127.54 & 65.93 & 21.87 & 12.13 & 4.84 \\
\hline
420 & 138.09 & 77.04 & 23.34 & 13.28 & 5.16 \\
\hline
440 & 150.81 & 77.11 & 25.22 & 13.84 & 5.84 \\
\hline
460 & 163.94 & 85.78 & 27.67 & 14.32 & 6.30 \\
\hline
480 & 177.77 & 95.85 & 28.10 & 15.50 & 7.15 \\
\hline
500 & 192.97 & 97.70 & 31.44 & 15.79 & 6.24 \\
\hline
520 & 209.49 & 108.34 & 32.95 & 17.27 & 6.62 \\
\hline
540 & 219.24 & 119.94 & 35.29 & 18.61 & 7.30 \\
\hline
560 & 237.04 & 125.29 & 36.40 & 18.91 & 7.54 \\
\hline
580 & 247.24 & 140.73 & 38.26 & 20.31 & 8.09 \\
\hline
600 & 270.45 & 147.62 & 40.02 & 20.88 & 8.90 \\
\hline
620 & 284.33 & 149.17 & 42.12 & 21.62 & 9.62 \\
\hline
640 & 309.54 & 166.93 & 43.10 & 22.76 & 9.35 \\
\hline
660 & 315.12 & 178.41 & 45.89 & 23.40 & 10.78 \\
\hline
680 & 342.33 & 188.27 & 48.52 & 25.07 & 11.13 \\
\hline
700 & 358.19 & 197.32 & 49.73 & 26.47 & 11.25 \\
\hline
720 & 373.96 & 215.76 & 51.21 & 26.79 & 12.10 \\
\hline
740 & 397.99 & 222.46 & 53.13 & 27.65 & 14.54 \\
\hline
760 & 411.69 & 232.85 & 54.71 & 29.08 & 13.95 \\
\hline
780 & 439.17 & 245.02 & 58.36 & 29.81 & 15.68 \\
\hline
800 & 461.42 & 264.66 & 59.94 & 31.70 & 15.90 \\
\hline
820 & 476.24 & 276.48 & 61.90 & 32.29 & 17.69 \\
\hline
840 & 496.48 & 285.15 & 64.12 & 34.03 & 18.62 \\
\hline
860 & 515.91 & 292.42 & 65.49 & 33.94 & 20.76 \\
\hline
880 & 539.34 & 309.45 & 68.73 & 35.85 & 22.05 \\
\hline
900 & 559.42 & 322.18 & 69.30 & 35.84 & 23.26 \\
\hline
920 & 592.42 & 332.22 & 72.63 & 37.46 & 24.07 \\
\hline
940 & 616.82 & 353.43 & 74.38 & 38.75 & 25.46 \\
\hline
960 & 645.81 & 372.20 & 76.63 & 40.61 & 27.16 \\
\hline
980 & 668.59 & 385.35 & 80.79 & 40.99 & 27.79 \\
\hline
1000 & 687.16 & 403.35 & 82.07 & 41.39 & 30.73 \\
\hline 
\end{tabular} 
\end{center}
\end{table}
\end{samepage}

\chapter{Referências Bibliográficas}

\begin{itemize}
\item \begin{verbatim}http://en.wikipedia.org/wiki/Bubble_sort\end{verbatim}
\item \begin{verbatim}http://en.wikipedia.org/wiki/Insertion_sort\end{verbatim}
\item \begin{verbatim}http://www.princeton.edu/~achaney/tmve/wiki100k/docs/Insertion_sort.html\end{verbatim}
\item \begin{verbatim}http://en.wikipedia.org/wiki/Merge_sort\end{verbatim}
\item \begin{verbatim}http://www.princeton.edu/~achaney/tmve/wiki100k/docs/Merge_sort.html\end{verbatim}
\item \begin{verbatim}http://en.wikipedia.org/wiki/Heapsort\end{verbatim}
\item \begin{verbatim}http://pages.cs.wisc.edu/~paton/readings/Old/fall01/HEAP-SORT.htm\end{verbatim}
\item \begin{verbatim}http://en.wikipedia.org/wiki/Quicksort\end{verbatim}
\item \begin{verbatim}http://www.personal.kent.edu/~rmuhamma/Algorithms/MyAlgorithms/Sorting/quickSort.htm\end{verbatim}
\end{itemize}


\end{document}
